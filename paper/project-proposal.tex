\documentclass[conference]{IEEEtran}

\usepackage{cite}
\usepackage{amsmath,amssymb,amsfonts}
\usepackage{mathpartir}
\usepackage{algorithmic}
\usepackage{graphicx}
\usepackage{textcomp}
\usepackage{xcolor}
\usepackage{url}
\usepackage{hyperref}
\def\BibTeX{{\rm B\kern-.05em{\sc i\kern-.025em b}\kern-.08em
    T\kern-.1667em\lower.7ex\hbox{E}\kern-.125emX}}
\begin{document}

\title{A New Functional Language with Effect Types and Substructural Handlers}

\author{
    \IEEEauthorblockN{Pavel Sokolov}
    \IEEEauthorblockA{
        \textit{Faculty of Computer Science} \\
        \textit{Higher School of Economics} \\
        Moscow, Russia \\
        ppsokolov@edu.hse.ru
    }
}

\maketitle

\begin{abstract}
We propose a new programming language where algebraic effects are visible to the
type system as polymorphic effect types. A substructural type system guarantees
a predictable behavior of effectful computations; we model these using row
polymorphism and recursive types. The result of the transformation lets us
represent effectful programs in the continuation-passing style, which should be
familiar to those who are acquainted with Freer monads. As a proof of concept,
we will implement an interpreter in Haskell, which we consider to be one of the
main inspirations for this project, with a self-hosting compiler on the way.
\end{abstract}

\begin{IEEEkeywords}
algebraic effects, substructural type system, row polymorphism,
equirecursive types
\end{IEEEkeywords}

\section{Introduction}

This project belongs to the area of programming language design. We aim to
develop a new functional language featuring a modern type system for ergonomic
management of side effects and program resources.

Simply put, side effects are everything a program does to the environment:
device management, reads from / writes to files, missile launches, etc.
Programming errors in this area range from embarrassing bugs to outright
catastrophes; every substantial development team wants to be sure that these
errors never happen. However, existing mainstream programming techniques such as
testing and code review are unsound, which means that you can never be sure that
there are no bugs even if all tests have passed.

Luckily, there is a sound method known to every seasoned programmer: type system
expressive enough to statically represent invariants you wish to uphold. In this
project, we implement a new language with effect types representing computations
that influence its environment. The idea is not novel: there are research
languages Unison \cite{unison} and Koka \cite{koka} on this topic; a simpler and
somewhat more tractable model is studied in \cite{bauer}. Our innovation is in
the usage of a substructural type system to statically verify that continuations
in effect handlers are called an appropriate number of times. A concept for such
combination is folklore and mentioned in \cite{folklore}; this project properly
expands on it.

The expected results are twofold. Firstly, we will present a profound design of
a sound substructural type system for a functional language with algebraic
effects. Secondly, we will provide a proof-of-concept interpreter in Haskell.
Formal verification of soundness and implementation of efficient self-hosting
compiler are reserved for future work.

We begin with a general analysis of effect systems including the ones mentioned
above as well as their analogues in Haskell. Then we overview the features in
our language that aid algebraic effects. Before concluding, we provide a basic
description of the type system and core typing rules.

\section{Literature Review}

The point of pure functional programming is to give a programmer precise control
over side effects that happen during program execution. The consensus is to use
special kinds of monads, notable examples being monad transformers \cite{mtl},
free monads \cite{free} and, most recently, freer monads. The last ones in
conjunction with heterogeneous lists give rise to the Extensible Effects
approach \cite{exteff} which, as claimed by the authors, overcomes the drawbacks
of previous tools. However, these techniques are rarely used outside of Haskell
because they require quite advanced type-level machinery which is simply not
available in other, either more simple or mainstream, languages.

Albeit surprisingly, the key concept behind Extensible Effects is simple. It is
to view an effect as an interaction between a sub-expression and a central
authority that handles effect execution. Languages with algebraic effects, for
their part, capitalize on this idea and provide built-in language primitives to
build both such sub-expressions --- from now on, we will call them
``procedures'' --- and central authorities, ``handlers''. Communication between
them is done via continuation-passing \cite{algeff}.

For example, Multicore OCaml has a way to declare new effects as well as a
builtin function \verb|perform| to, indeed, \textit{perform} an effect
\cite{ocaml}. Handler is just a regular \verb|case| statement that
pattern-matches on performed effects. However, for a long time effects were
invisible to OCaml's type system; this means that every function could be a
procedure executing arbitrary effects, which is in no way different from
the mainstream approach.

As for Unison, they introduce proper effect types in the form of attributes on
the function type. However, they do not restrict how many times a continuation
is called in a handler \cite{unrestricted}, which may lead to bizarre and
unexpected behavior of a program, just like a program in C which uses
\verb|fork|. Koka's treatment of continuations is only slightly better as they
do not let a programmer access continuation directly, which enables them (and
makes them) to introduce a whole lot of ad-hoc mechanisms to make both behavior
and performance of programs more predictable \cite{hidden}.

A more disciplined solution is offered in \cite{bauer}. The authors of Runners
in Action simplify the framework by forcing continuations to be in tail-call
positions, effectively replacing continuation with a simple \verb|return|
statement (pun intended). However, it makes certain effects inexpressible, most
notably nondeterminism and asynchrony.

\section{Methodology}

\subsection{General description}

We get rid of the tradeoffs mentioned above by embracing a substructural type
system. In its setting, every value can have two constraints imposed on its
usage: can it be silently dropped? Can it be copied? We simply make a programmer
specify the constraints for continuations while declaring new effects. As a
bonus, these constraints can be later used by an optimizing compiler to place
values outside of the garbage-collected heap.

Actually, there already are a few different substructural type systems.
Linear Types extension in Haskell \cite{linear} and unique types in Rust
\cite{rust} are the most well-known examples, although we take a slightly
different approach which is extensively described in \cite{ural}. As of our
current knowledge, this project is the first to utilize URAL for language with
algebraic effects.

Let us now pay attention to the ergonomics of the language. Effect types
themselves make type signatures more complicated than those in mainstream
languages; to make matters worse, linear types are notorious for their brimming
function signatures. This calls for global type inference. To separate concerns
about type inference and syntactic sugar from the rest of the language, we
followed in the footsteps of Glasgow Haskell Compiler and split the language in
two parts \cite{ghc}: surface language for developers to write programs in and
core language for type checking and evaluation algorithms to run in.

In order to encode inductive data types of a surface language, we, perhaps
mistakingly, decided to use equirecursive types as they are defined in
\cite{stone}. However, we discovered that this opens a possibility to eliminate
algebraic effects from the core language altogether: coupled with row
polymorphism like in \cite{rowkoka} and \cite{rowcaml}, equirecursive types
allow us to desugar types of computations into open unions where each variant
represents one effectful operation; this encoding is very similar to Freer
monads \cite{exteff}. After employing this encoding, effect handlers become
nothing more than recursive functions which accept computations and
pattern-match on them. Nice!

One should note, though, that this encoding is not the only reason why we
incorporated row types. It simplifies core language: without it, one would
typically need to introduce a binary operation on types and unit of it for every
kind of composite data structure (multiplicative conjunction, additive
conjunction, additive disjunction), which, together with the introduction and
elimination rules, would comprise a whole lot of logical rules. With our
generalized flavor of row types, we needed only three rules for each kind; the
idea of generalization itself stems from the similarities between records for
both multiplicative and additive conjunctions.

\subsection{Typing rules}

Having reached a general understanding of our type system, let us describe
typing rules of a core language.

\subsubsection{Kinds}

Our core language has the following kinds:
\[
    \begin{array}{rl}
        L ::=& \text{pretype} \;|\; \text{type} \;|\; L \times L \\
        K ::=& L \;|\; \text{mult} \;|\; \text{row} \; K \;|\; K \times K
            \;|\; K \to K \\
    \end{array}
\]

Kinds in $L$ are called ``simple kinds'', whereas kinds in $K$ are ``proper
kinds'' or just ``kinds''.

Types are pretypes annotated with multiplicity modality (terms of kind mult).
$\times$ is a kind constructor for type-level pairs, which is extremely useful
for both mutually recursive types and desugaring of effect types. Note that
row types have another kind as a parameter: they are used to represent a labeled
collection of any kind, not only of types.

\subsubsection{Type-level operators}

\begin{figure}
    \centering
    \begin{mathpar}
        \inferrule
            {k \; \text{kind}}
            {K,\; a :: k \;\vdash\; a :: k}
            \textsc{KVar} \and
        \inferrule
            {k, k' \; \text{kind} \\ K \;\vdash\; x :: k
                \\ K \;\vdash\; t \;::\; k \to k'}
            {K \;\vdash\; t \; x :: k'}
            \textsc{KApp} \and
        \inferrule
            {k, k' \; \text{kind} \\ K,\; a :: k \;\vdash\; b :: k'}
            {K \;\vdash\; (a :: k .\; b) :: k \to k'}
            \textsc{KAbs} \and
        \inferrule
            {K \;\vdash\; a :: k \\ K \;\vdash\; b :: k'}
            {K \;\vdash\; \langle a, b \rangle :: k \times k'}
            \textsc{KPair} \and
        \inferrule
            {K \;\vdash\; p \;::\; k \times k'}
            {K \;\vdash\; \text{fst} \; p :: k}
            \textsc{KFst} \and
        \inferrule
            {K \;\vdash\; p \;::\; k \times k'}
            {K \;\vdash\; \text{snd} \; p :: k'}
            \textsc{KSnd} \and
        \inferrule
            {k \; \text{simple kind} \\ K,\; a :: k \;\vdash\; b :: k}
            {K \;\vdash\; (\mu\; a.\; b) :: k}
            \textsc{KFix} \and
        \inferrule
            {k, k' \; \text{kind} \\ K \;\vdash\; r :: \text{row} \; k
                \\ K \;\vdash\; t \;::\; k \to k'}
            {K \;\vdash\; t \;@\; r \;::\; \text{row} \; k'}
            \textsc{KMap}
    \end{mathpar}
    \caption{Type-level operations}
    \label{typelevel}
\end{figure}

Rules for type-level operations are presented in Fig.\ref{typelevel}. Kinding
context is supposed to be a usual context supporting exchange, weakening, and
contraction. Once we introduce type-level operations, we also have to determine
operational semantics (which are fairly obvious) and sound equivalence
relations. This is the whole point of \cite{stone}; we only have to extend
algorithms to handle polymorphic multiplicities and rows, which are elaborated
further. General type-level expressions are later denoted as $E$.

\subsubsection{Multiplicities}

Multiplicities form a distributive lattice on four elements:
\[
    \begin{array}{rl}
        M ::=& E \;|\; ! \;|\; ? \;|\; + \;|\; * \;|\; M \lor M
            \;|\; M \land M
    \end{array}
\]

Multiplicity literals draw an analogy with the notation of regular expressions.
$\lor$ is used to join constraints in composite structures; $\land$ is used to
desugar constraints on multiplicity variables. For example, a type of function
duplicating its argument is desugared from the surface language like this:
\begin{multline}
    [m \le +] \to a^m \to (a^m, a^m) \quad\rightsquigarrow\\
    \big(\text{Forall} (a :: \text{pretype}) (m :: \text{mult}) .\\
    (a^{m \land +} \to \{.0: a^{m \land +}, .1: a^{m \land +}\})^*\big)^*
\end{multline}

General expressions for multiplicities are strongly normalizing, therefore we
can evaluate them eagerly. The subsumption of multiplicities can be decided in
quadratic time by a straightforward recursive algorithm. Equality of
multiplicities can be reduced to checks of two subsumptions.

\subsubsection{Row types}

Rows form a join semilattice:
\[
    \begin{array}{rl}
        R ::=& E \;|\; \varnothing \;|\; (.n_j : t_j, ... R)
            \;|\; R \lor R
    \end{array}
\]

One should note that, because of the substructural type system, we cannot
silently drop repeating entries as is done in JavaScript. Therefore rows can
have duplicated labels; however, we provide no disambiguation between duplicated
entries because this would not respect the absence of order between them.

Checking the equality of two rows is not different from deciding the equality of
two multimaps.

\subsubsection{(Pre)types}

Pretypes contain functions, forall binders and composite types ($R$ below are
assumed to be rows of types):
\[
    \begin{array}{rl}
        T ::=& E \;|\; P^M \\
        P ::=& E \;|\; T \to T \;|\; \Pi (x :: K). T \\
        \;|\;& \{...R\} \;|\; [...R] \;|\; (|...R|)
    \end{array}
\]

We do not expect any complications while adapting the algorithm for checking
type equality from \cite{stone}.

\subsubsection{Modelling effect types in the core language}

Now, we are ready to define the algebra of effect types as well as desugaring
for their expressions. The algebra looks very similar to the algebra of row
types:
\[
    \begin{array}{rl}
        \text{Eff} ::=& E \;|\; \varnothing
            \;|\; (op : T \xrightarrow{M} T, ...\text{Eff})
            \;|\; \text{Eff} \lor \text{Eff}
    \end{array}
\]

In addition, just like with row types, effect types can contain duplicate
entries. In \cite{rowkoka} it is argued that, actually, this is a good thing.

Let us denote the type of computation producing effects in $e$ and returning
value of type $a$ as $e \models a$. The desugaring procedure goes as follows:
\begin{equation}
    \begin{array}{rcl}
        e &\rightsquigarrow& e' \\
        f \; t &:=& \langle u, m, v \rangle \Rightarrow
            (u, v \xrightarrow{m} t)^{\text{mult}\;u \;\lor\; m} \\
        e \models a &\rightsquigarrow& \mu t.
            (| \text{.pure} : a, ...(f \; t \; @ \; e') |)
    \end{array}
\end{equation}

As expected from the similarity with rows, effect types are first desugared into
rows of triples: $\text{type} \times \text{mult} \times \text{type}$.

\subsubsection{Terms}

\begin{figure}
    \centering
    \begin{mathpar}
        \inferrule
            {K;\; \Gamma,\;x:t,\;y:u,\;\Delta \;\vdash\; e:T}
            {K;\; \Gamma,\;y:u,\;x:t,\;\Delta \;\vdash\; e:T}
            \textsc{ $\Gamma$-Exchange} \and
        \inferrule
            {m \le\; ? \\ K;\; \Gamma \;\vdash\; e:T}
            {K;\; x:p^m,\; \Gamma \;\vdash\; e:T}
            \textsc{ $\Gamma$-Weakening} \and
        \inferrule
            {m \le + \\ K;\; x:t^m,\; x:t^m,\; \Gamma \;\vdash\; e:T}
            {K;\; x:t^m,\; \Gamma \;\vdash\; e:T}
            \textsc{ $\Gamma$-Contract} \and
        \inferrule{t \; \text{type}}{K;\; x:t \;\vdash\; x:t} \textsc{TVar} \and
        \inferrule
            {K;\; \Gamma,\; x:t \;\vdash\; e:u
                \\ K \;\vdash\; \text{sup} \; \Gamma \le m}
            {K; \Gamma \;\vdash\; (\lambda x.\; e) : (t \to u)^m}
            \textsc{TAbs} \and
        \inferrule
            {K;\Gamma \;\vdash\; f : (t \to u)^m \\ K;\Delta \;\vdash\; x : u}
            {K;\; \Gamma, \Delta \;\vdash\; f \; x : u}
            \textsc{TApp} \and
        \inferrule
            {a :: k,\; K;\; \Gamma \;\vdash\; e:t
                \\ K \;\vdash\; \text{sup} \; \Gamma \le m}
            {K;\Gamma \;\vdash\; (\Lambda a :: k.\; e)
                : \left(\Pi(a :: k). t\right)^m}
            \textsc{TGen} \and
        \inferrule
            {K;\Gamma \;\vdash\; f : \left(\Pi(a :: k). t\right)^m
                \\ K \;\vdash\; d :: k}
            {K;\Gamma \;\vdash\; f \; d : \{a / d\} t}
            \textsc{TInst} \and
        \inferrule
            {\forall n \in\text{keys}\;R: \quad K;\Gamma_n \;\vdash\; e_n : R[n]
                \\\\ K \;\vdash\; \text{sup} \; R \le m}
            {K;\Gamma_n \;\vdash\; \{.n : e_n\} : \{\dots R\}^m}
            \textsc{TAndI} \and
        \inferrule
            {K;\Gamma \;\vdash\; s : \{\dots R\}^m
                \\ K;\; \Delta,R \;\vdash\; e:t}
            {K;\;\Gamma,\Delta \;\vdash\;
                \textbf{let} \; \{.1,\dots,.n\} = s \; \textbf{in} \; e : t}
            \textsc{TAndE} \and
        \inferrule
            {\forall n \in \text{keys}\;R: \quad K;\Gamma \;\vdash\; e_n : R[n]
                \\\\ K \;\vdash\; \text{sup} \; R \le m}
            {K;\Gamma \;\vdash\; [.n: e_n] : [\dots R]^m}
            \textsc{TWithI} \and
        \inferrule
            {K;\Gamma \;\vdash\; w : [\dots R]^m \\ n \in \text{keys} \; R}
            {K;\Gamma \;\vdash\; w.n : R[n]}
            \textsc{TWithE} \and
        \inferrule
            {K;\Gamma \;\vdash\; e : t^{m'} \\ K \;\vdash\; m' \le m
                \\ n \in \text{keys} \; R}
            {K;\Gamma \;\vdash\; .n \; e : (|\dots R|)^m}
            \textsc{TOrI} \and
        \inferrule
            {K;\Gamma \;\vdash\; o : (|\dots R|)^m
                \\\\ \forall n \in \text{keys}\;R:
                \quad K;\; x:R[n],\; \Delta \;\vdash\; e_n : t}
            {K;\;\Gamma,\Delta \;\vdash\; \textbf{case} \; o
                \; \textbf{of} \; (.n x := e_n) : t}
            \textsc{TOrE}
    \end{mathpar}
    \caption{Context and Terms}
    \label{terms}
\end{figure}

Rules for terms follow straightforwardly from rules for (pre)types and are
provided in Fig.\ref{terms}. Rules for substructural term context are provided
in the same figure. Note that while defining operations on rows, we have to
account for polymorphic rows, though it is quite simple.

\section{Results}

It is expected to will have implemented an interpreter of the language described
above, including the split in the surface and core languages: logical rules of
the core language are already properly written out, and the Haskell ecosystem
has all the necessary tools.

Type inference and the existence of principal types in the surface language are
highly anticipated because both row types and substructural types play nicely
with Hindley-Milner. However, general equirecursive types may pose a threat to
the type inference, therefore they might be hidden beneath the surface.

\section{Conclusion}

Utilising the recent advancements in type theory, we developed a polished model
of computations with side effects which is an improvement upon analogues in
terms of soundness and, theoretically, performance. While doing so, we made
initially unrelated constructs from different type systems interplay nicely: row
polymorphism with substructural typing, both type-level pairs and row
polymorphism with algebraic effects... In addition, we saved ourselves the
headache of developing a dedicated runtime for algebraic effects using a
well-typed encoding into continuation-passing style.

The most notable limitation of our solution is that core language is hardly
extensible: generalized algebraic data types cannot be easily fit into it;
coexistence of dependent types with substructural types is still an ongoing
research topic;... To overcome this limitation, we can make use of the
two-layered architecture of the language and change core language independently
from the surface to incorporate new features.

Nevertheless, as it was said in the beginning, the primary goals after this
project are to formally verify the soundness of the type system and to write a
self-hosting compiler. The first one is needed primarily to exclude the human
factor; writing a compiler is required to fulfill promises of performance.

Designing one more programming language might not change anything in the grand
scheme of things; however, in order to reach the ideal future where every
programmer writes programs in languages with effect types, we as a whole
programming community need enough failed attempts to learn from and working
alternatives to choose from.

\bibliographystyle{IEEEtran}
\bibliography{project-proposal}

\end{document}
